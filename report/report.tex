\documentclass{article}
\begin{document}
\section{Behavior for small energies}
For small energies the density of states does not have the same smooth behavior as at large energies.
This is because there are only a small number of possible ways (up to translation, rotation, reflection, and relabelling of spins) of obtaining each of these energies.

As an example, for lattices of size at least 2, there are $q$ $E = 0$ states, $0$ states for $E \in \{1,2,3,5}$, but $q^2$ states for $E = 4$.
The case for $E = 1$ is shown here.

Suppose that there was a state with $E = 1$.
Then one particle differs from an adjacent one.
We can assume without loss of generality that this particle is at position $(1,1)$, and it has a different spin from the particle at $(0,1)$.
But then $(1,1)$ must have the same spin as $(1,0)$, which has the same spin as $(0,0)$, which has the same spin as $(0,1)$.
So this is impossible.
\end{document}
